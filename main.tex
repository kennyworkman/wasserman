\documentclass[10pt]{article}
\usepackage{kennyworkman}

\title{Wasserman: All of Statistics}
\author{Kenny Workman}
\date{\today}

\begin{document}

\maketitle

\section{Probability}

\subsection{Basics}

\begin{example}

It's actually tricky to show $\P(A \cup B) = \P(A) + \P(B) - \P(A \cap B)$ using
only the three axioms:

\begin{align*}
\P(A \cup B) &= \P(AB^c \cap AB \cap A^cB) \\
&= \P(AB^c) + \P(AB) + \P(A^cB) \\
&= \P(AB^c) + \P(AB) + \P(A^cB) + P(AB) - P(AB) \\
&= \P(AB^c \cup AB) + \P(A^cB \cup AB) - P(AB) \\
&= \P(A) + \P(B) - P(AB)
\end{align*}
\end{example}

Another simple idea is that events that are identical at the limit should have
identical probabilities.

\begin{theorem}[Continuity of Events]
If $A_n \to A$ then $\P(A_n) \to \P(A)$.
\end{theorem}

\begin{proof}
    Let $A_n$ be monotone increasing: $A_1 \subset A_2 \subset \dots$. Let
    $A = \lim_{n \to \infty} A_n = \bigcup_{i=1}^{\infty} A_i$.

    Construct disjoint sets $B_i$ from each $A_i$ where $B_1 = A_1$ and $B_n =
    \{ \omega \in \Omega : \omega \in A_n,\omega \notin \bigcup_{i=1}^{i-1}A_i \}$. It will be
    shown that (1) each pair of $B_i$ are disjoint, (2) $\bigcup_{i=1}^n A_i =
    \bigcup_{i=1}^n B_i$ and (3) $A = \bigcup_{i=1}^\infty A_i =
    \bigcup_{i=1}^\infty B_i$ (Exercise 1.1).

    From Axiom 3: $\P(A_n) = \P(\bigcup\limits_{i=1}^n A_i) = \P(\bigcup\limits_{i=1}^n B_i) =
    \sum\limits_{i=1}^n \P(B_i)$.

    Then $\lim_{n \to \infty} \P(A_n)
    = \lim_{n \to \infty} \sum\limits^{n}\P(B_n)
    = \sum\limits^{\infty}\P(B_n)
    = \P(\bigcup\limits^{\infty}B_n)
    = \P(A)$

\end{proof}

\begin{exercise}
    Fill in the details for Theorem 1.2 and extend to the case where $A_n$ is monotone decreasing.
\end{exercise}

\begin{proof}
    For any pair $B_{n+1}$ and $B_n$, because $B_n \subset A_n$ and $B_{n+1}
    \cap A_n = \emptyset$, it follows that $B_{n+1} \cap B_n = \emptyset$.

    Let $\bigcup_{i=1}^n B_i = \bigcup_{i=1}^n A_i$. Then $\bigcup_{i=1}^{n+1}
    B_i = (A_{n+1} \setminus \bigcup_{i=1}^n A_i) \bigcup (\bigcup_{i=1}^n
    A_i) = \bigcup_{i=1}^{n+1} A_i$.

    For the monotone decreasing case, let $A_n$ be a sequence where $A_1 \supset A_2 \supset A_3 \dots$. 

    Observe $A_1^c \subset A_2^c \dots$ and $\lim_{n \to \infty} A_n = \Omega
    \setminus \bigcup^{\infty} A_i^c$. Construct disjoint $B^c_n$ from $A^c$ in
    the same way.

    Then $\lim_{n \to \infty} \P(A_n) = 1 - \sum\limits^{\infty} \P(B^c_i) = 1 -
    \P(A^c) = \P(A)$
\end{proof}


\setcounter{exercise}{2}
\begin{exercise}
    Let $\Omega$ be a sample space and $A_1, A_2, \dots$ be events. Define
    $B_n = \cup_{i=n}^{\infty} A_i$ and $C_n = \cap_{i=n}^{\infty} A_i$.
    \begin{enumerate}[(a)]
        \item{Show $B_1 \supset B_2 \supset B_3 \dots$ and $C_1 \subset C_2
            \subset C_3 \dots$}
        \item{Show $\omega \in \cap_{n=1}^{\infty} B_n$ iff $\omega$ is in an
            infinite number of the events}
        \item{Show $\omega \in \cup{n=1}^{\infty} C_n$ iff $\omega$ belongs to all
                of the events, except possibly a finite number of those
            events.}
    \end{enumerate}
\end{exercise}

\begin{proof}
    \begin{enumerate}[(a)]
        \item{Certainly $\cup_{i=1}^{\infty} A_i \supset \cup_{i=2}^{\infty} A_i \dots$ and $\cap_{i=1}^{\infty} A_i \subset \cap_{i=2}^{\infty} \dots$.}
        \item{Forward. Assume $\omega \in \cap_{n=1}^{\infty}B_n$. If $\omega$
            does not belong to an infinite number of events $A_i$, there exists
        some index $j$ past which $\omega \notin B_j$. Then certainly $\omega \notin \cap_{n=1}^{\infty}B_n$. Reverse. $\omega$ belonging to infinite events means there cannot exist such a $j$ described previously so $\omega \in B_n$ for all $n$. Indeed $\omega \in \cap_{n=1}^{\infty}B_n$}
    \item{Forward. Assume $\omega \in \cup_{n=1}^{\infty}C_n$. Then $\omega \in
        C_j = \cap_{i=j}^{\infty} A_i$ for some $j$. This is another way of
    saying $\omega$ is in every single event except for perhaps a finite number
in $A_{i < j}$. Reverse. Let $j$ be the index of the largest event that
$\omega$ is not in. Then $\omega \in C_{n > j}$ and certainly $\omega \in
\cup^{\infty}C_n$.}
    \end{enumerate}
\end{proof}

\begin{note}
    The key idea above is this notion of "infinitely often" (i.o.) and "all but finitely
    often" (eventually) which are two distinct structures of infinite occurence in
    sequences. Consider an $\omega$ that exists in every other event (eg. just the
    odd indices) for infinite events and revisit its inclusion in $\cap^{\infty}
    B_i$ and $\cup^{\infty} C_i$.
\end{note}

\begin{note}
    % https://chatgpt.com/c/67c5f3d7-48f4-8004-a28f-8f25f47b44c4
    $\lim \cap \cup A_n$ is also referred to as the limit infimum of $A_n$. Similarly,
    $\lim \cup \cap A_n$ is referred to as the limit supremum of $A_n$.
    %TODO: show equivalence
    % Then revisit Borel Cantelli lemma: https://en.wikipedia.org/wiki/Borel%E2%80%93Cantelli_lemma
\end{note}

\setcounter{exercise}{6}
\begin{exercise}
    Let $\P(\bigcup\limits^n A_i) \leq \sum\limits^n \P(A_i)$. Then
    $\P(A_{n+1} \cup (\bigcup\limits^n A_i)) \leq \P(A_{n+1}) +
    (\sum\limits^n \P(A_i)) - \P(A_{n+1} \cap(\bigcup\limits^n A_i)) \leq
    \sum\limits^{n+1}\P(A_i)$
\end{exercise}

\begin{note}
% https://en.wikipedia.org/wiki/Boole%27s_inequality
    Expand a bit on the Boole inequality.
\end{note}
\end{document}
